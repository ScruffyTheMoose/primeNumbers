\documentclass[11pt]{article}
\usepackage[paperwidth=10.2in,paperheight=11.7in]{geometry}


\usepackage{framed}

\usepackage{amsmath, amssymb}
\usepackage{listings}
\usepackage{color}
\usepackage{graphicx}

\setlength{\parindent}{0pt}
\setlength{\parskip}{12pt}
\setlength{\voffset}{-30pt}
\setlength{\hoffset}{-40pt}
\setlength{\textwidth}{600pt}
\setlength{\textheight}{720pt}
\setlength{\marginparwidth}{5pt}

\definecolor{codegreen}{rgb}{0,0.6,0}
\definecolor{codegray}{rgb}{0.5,0.5,0.5}
\definecolor{codepurple}{rgb}{0.58,0,0.82}
\definecolor{backcolour}{rgb}{0.95,0.95,0.92}

\lstdefinestyle{Theta}{
	backgroundcolor = \color{backcolour},
	commentstyle = \color{codegreen},
	keywordstyle = \color{magenta},
	numberstyle = \tiny\color{codegray},
	stringstyle = \color{codepurple},
	basicstyle = \ttfamily\footnotesize,
	breakatwhitespace = false,         
	breaklines = true,                 
	captionpos = b,                    
	keepspaces = true,                 
	numbers = left,                    
	numbersep = 6pt,                  
	showspaces = false,                
	showstringspaces = false,
	showtabs = false,                  
	tabsize = 2,
	language = Python,
	frame = single,
	title = {Algorithm for calculating $\theta$(x) for a list of prime numbers using Chebyshev's first function, returning a list of all $\theta$(x) values in ascending order.}
}	
\begin{document}
	
\lstset{style = Theta}

\begin{lstlisting}
	
	from math import log
	
	
	def run(primeList: list) -> list:
		"""
		Chebyshev's Theta Function.
		Returns a sorted list containing the log transformed product of the primorial at each prime in the given list.
		"""
		
		# list for storing the product at each nth prime
		products = list()
		
		# initiating var to store current Theta(x)
		lastVal = 0
		
		# iterating through all primes in the list
		for prime in primeList:
		
			# getting sum of logs
			# using log laws, we know log(n) + log(m) == log(nm)
			current = log(prime) + lastVal
			
			# storing to list
			products.append(current)
			
			# updating product
			lastVal = current
		
		return products
	
		
\end{lstlisting}


Chebyshev's First Function is defined as

$\theta(x) = \sum_{\textit{p} \leq x} \log(\textit{p})$

where \textit{p} is a prime number.

Because this function is a sum, the provided list of primes must include all primes less than or equal to the greatest prime in the given list. If all primes are not included, the resulting $\theta$(x) will not be accurate.

Because this is an arithmetic sequence, for each \textit{p}, we take its natural log and add it to the previously determined $\theta$(x).

The end result is a list of values for $\theta$(x).

Interestingly, the resulting curve produced by this function acts as a lower bound to the set of prime numbers which it represents. The curve is also log-linear, which allows for very loose approximation of where a prime number may fall on $\mathbb{Z}$.

The relative difference between $\theta(x)$ and x approaches 0 as x tends to infinity.

\end{document}