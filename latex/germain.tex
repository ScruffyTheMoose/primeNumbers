\documentclass[11pt]{article}
\usepackage[paperwidth=10.2in,paperheight=11.7in]{geometry}


\usepackage{framed}

\usepackage{amsmath, amssymb}
\usepackage{listings}
\usepackage{color}
\usepackage{graphicx}

\setlength{\parindent}{0pt}
\setlength{\parskip}{12pt}
\setlength{\voffset}{-30pt}
\setlength{\hoffset}{-40pt}
\setlength{\textwidth}{600pt}
\setlength{\textheight}{720pt}
\setlength{\marginparwidth}{5pt}

\definecolor{codegreen}{rgb}{0,0.6,0}
\definecolor{codegray}{rgb}{0.5,0.5,0.5}
\definecolor{codepurple}{rgb}{0.58,0,0.82}
\definecolor{backcolour}{rgb}{0.95,0.95,0.92}

\lstdefinestyle{Germain}{
	backgroundcolor = \color{backcolour},
	commentstyle = \color{codegreen},
	keywordstyle = \color{magenta},
	numberstyle = \tiny\color{codegray},
	stringstyle = \color{codepurple},
	basicstyle = \ttfamily\footnotesize,
	breakatwhitespace = false,         
	breaklines = true,                 
	captionpos = b,                    
	keepspaces = true,                 
	numbers = left,                    
	numbersep = 6pt,                  
	showspaces = false,                
	showstringspaces = false,
	showtabs = false,                  
	tabsize = 2,
	language = Python,
	frame = single,
	title = {Algorithm to identify sequences of Germain primes.}
}	
\begin{document}
	
\lstset{style = Germain}

\begin{lstlisting}
	
	def run(primes: list) -> dict:
		"""
		Returns a dict of all Germain Prime sequences identified in the given list
		"""
	
		# dict for storing results
		sequences = dict()
		
		# building set of primes of O(1) checking
		primeSet = set(primes)
		
		# iterating through all primes in given list
		for prime in primes:
		
			# list for storing the current sequence achieved
			seq = list()
			
			# assigning the first prime to check as the current prime in the given list of primes
			gt = prime
			
			# checking that gt is Germain, and if so, adding to sequence and updating gt
			while (gt * 2) + 1 in primeSet:
				seq.append(gt)
				
				gt = (gt * 2) + 1
		
		# if the seq variable is not empty, meaning at least one Germain prime was identified, it gets added to results
		if seq:
			sequences[prime] = {"sequence": seq, "length": len(seq)}
		
		# returning the results
		return sequences

		
\end{lstlisting}


Let \textit{p} be a prime number.

\textit{p} is a Germain Prime if $2\textit{p}+1$ is also prime.



\end{document}