\documentclass[11pt]{article}
\usepackage[paperwidth=10.2in,paperheight=11.7in]{geometry}


\usepackage{framed}

\usepackage{amsmath, amssymb}
\usepackage{listings}
\usepackage{color}
\usepackage{graphicx}

\setlength{\parindent}{0pt}
\setlength{\parskip}{12pt}
\setlength{\voffset}{-30pt}
\setlength{\hoffset}{-40pt}
\setlength{\textwidth}{600pt}
\setlength{\textheight}{720pt}
\setlength{\marginparwidth}{5pt}

\definecolor{codegreen}{rgb}{0,0.6,0}
\definecolor{codegray}{rgb}{0.5,0.5,0.5}
\definecolor{codepurple}{rgb}{0.58,0,0.82}
\definecolor{backcolour}{rgb}{0.95,0.95,0.92}

\lstdefinestyle{MillerRabin}{
	backgroundcolor = \color{backcolour},
	commentstyle = \color{codegreen},
	keywordstyle = \color{magenta},
	numberstyle = \tiny\color{codegray},
	stringstyle = \color{codepurple},
	basicstyle = \ttfamily\footnotesize,
	breakatwhitespace = false,         
	breaklines = true,                 
	captionpos = b,                    
	keepspaces = true,                 
	numbers = left,                    
	numbersep = 6pt,                  
	showspaces = false,                
	showstringspaces = false,
	showtabs = false,                  
	tabsize = 2,
	language = Python,
	frame = single,
	title = {Python module for running the Miller-Rabin primality test. Will return boolean value to indicate if the number being testing is composite or \emph{probably} prime. Optimal t-value is 60 with probability of failure being approximately $2^{-128}$.}
}

\lstdefinestyle{ChebyshevTheta}{
    backgroundcolor = \color{backcolour},
    commentstyle = \color{codegreen},
    keywordstyle = \color{magenta},
    numberstyle = \tiny\color{codegray},
    stringstyle = \color{codepurple},
    basicstyle = \ttfamily\footnotesize,
    breakatwhitespace = false,         
    breaklines = true,                 
    captionpos = b,                    
    keepspaces = true,                 
    numbers = left,                    
    numbersep = 6pt,                  
    showspaces = false,                
    showstringspaces = false,
    showtabs = false,                  
    tabsize = 2,
    language = Python,
    frame = single,
    title = {Python code to calculate $\theta$(x), also known as Chebyshev's First Function.}
}

\lstdefinestyle{GermainSequence}{
	backgroundcolor = \color{backcolour},
	commentstyle = \color{codegreen},
	keywordstyle = \color{magenta},
	numberstyle = \tiny\color{codegray},
	stringstyle = \color{codepurple},
	basicstyle = \ttfamily\footnotesize,
	breakatwhitespace = false,         
	breaklines = true,                 
	captionpos = b,                    
	keepspaces = true,                 
	numbers = left,                    
	numbersep = 6pt,                  
	showspaces = false,                
	showstringspaces = false,
	showtabs = false,                  
	tabsize = 2,
	language = Python,
	frame = single,
	title = {Python function to generate sequences of Germain Primes.}
}

\lstdefinestyle{Palindromes}{
	backgroundcolor = \color{backcolour},
	commentstyle = \color{codegreen},
	keywordstyle = \color{magenta},
	numberstyle = \tiny\color{codegray},
	stringstyle = \color{codepurple},
	basicstyle = \ttfamily\footnotesize,
	breakatwhitespace = false,         
	breaklines = true,                 
	captionpos = b,                    
	keepspaces = true,                 
	numbers = left,                    
	numbersep = 6pt,                  
	showspaces = false,                
	showstringspaces = false,
	showtabs = false,                  
	tabsize = 2,
	language = Python,
	frame = single,
	title = {Python function to identify palindromic prime numbers.}
}

\begin{document}

\lstset{style = MillerRabin}

\begin{lstlisting}

from random import randrange


def run(n: int, t: int) -> bool:
	"""
	n: number to be evaluated
	t: number of test iterations
	Function for running the Miller-Rabin primality test.
	Returns a boolean showing that n is either composite or probably prime.
	"""

	if n == 2 or n == 3:
		return True

	if n > 2 and n % 2 == 0:
		return False  # n is even
	
	# we will halve m iteratively until we achieve the equality:
	# n - 1 = (2^k)m
	k = 0
	m = n - 1
	
	# this loop halves m
	while m % 2 == 0:
		k += 1
		m //= 2  # floor div

	# we now have k, m such that m is an odd integer

	for _ in range(t):

		# determining random test value in range [2, n - 1]
		a = randrange(2, n - 1)
		
		# getting initial value for b
		b = pow(a, m, n)

		# initial check for primality
		if b == 1 or b == n - 1:
			continue

	# iterate until a result is found
	for _ in range(k - 1):

	# raising b**2 and getting remainder from modulo n
	b = pow(b, 2, n)
	
	# checking value of b
	if b == n - 1:
		break

	# if inner loop ends, check reason for ending
	# loop was broken
	if b == n - 1:
		continue
	# loop ran out
	else:
		return False

	# if all iterations were completed without throwing False, then n is probably prime
	return True

\end{lstlisting}

\newpage

\lstset{style = ChebyshevTheta}

\begin{lstlisting}

from math import log


def run(primeList: list) -> list:
	"""
	Chebyshev's Theta Function.
	Returns a sorted list containing the log transformed product of the primorial at each prime in the given list.
	"""

	# list for storing the product at each nth prime
	products = list()

	# initiating var to store current Theta(x)
	lastVal = 0

	# iterating through all primes in the list
	for prime in primeList:

		# getting sum of logs
		# using log laws, we know log(n) + log(m) == log(nm)
		current = log(prime) + lastVal
		
		# storing to list
		products.append(current)
		
		# updating product
		lastVal = current

	return products

\end{lstlisting}

\newpage

\lstset{style = GermainSequence}

\begin{lstlisting}

def run(primes: list) -> dict:
	"""
	Returns a dict of all Germain Prime sequences identified in the given list
	"""

	# dict for storing results
	sequences = dict()
	
	# building set of primes of O(1) checking
	primeSet = set(primes)
	
	# iterating through all primes in given list
	for prime in primes:

		# list for storing the current sequence achieved
		seq = list()
		
		# assigning the first prime to check as the current prime in the given list of primes
		gt = prime
		
		# checking that gt is Germain, and if so, adding to sequence and updating gt
		while (gt * 2) + 1 in primeSet:
		seq.append(gt)
		
		gt = (gt * 2) + 1
		
		# if the seq variable is not empty, meaning at least one Germain prime was identified, it gets added to results
		if seq:
		sequences[prime] = {"sequence": seq, "length": len(seq)}

	# returning the results
	return sequences

\end{lstlisting}

\newpage

\lstset{style = Palindromes}

\begin{lstlisting}

def run(prime: int) -> bool:
	"""
	Checks if the given prime is a palindrome arithmetically.
	Returns boolean.
	"""

	if prime < 10:
		return True

	# saving prime to variable n to check if n == reverse
	n = prime

	# variable to track reversed prime value
	rev = 0

	# since we are using floor division for prime, we iterate until prime <= 0
	while prime > 0:

		# the currect digit being worked on is the remainder from mod 10, giving us the last digit
		dig = prime % 10
		
		# multiply current reverse by 10 to allow for addition of dig
		rev = (rev * 10) + dig
		
		# floor division on prime to truncate last digit
		prime = prime // 10

	# if n, the starting prime, is equal to the reverse then it is a palindrome
	if n == rev:
		return True
	else:
		return False

\end{lstlisting}


\end{document}

